% !TeX program = xelatex

\documentclass{article}
\usepackage{xeCJK}

\setCJKmainfont{Adobe Song Std}

\begin{document}
	
\section{Reinforcement Learning}

\subsection{Reward function}

\subsection{Types of reward signals}

\subsubsection{Sparse Reward}

Curse of sparse reward.

许多有趣的问题最自然地被描述为稀疏的奖励信号 。在稀疏环境中寻找最优策略通常是不可行的,因为依赖非定向探索(undirected exploration)的通用算法需要“大得令人望而却步”的训练步数。

\subsubsection{Dense Reward}

促进学习(Facilitating Learning): 稀疏奖励可以通过基于势能的奖励塑造(potential-based reward shaping) 转换为密集奖励。这种转换旨在极大地促进学习过程。

连续逼近(Successive Approximation): 历史悠久的行为科学技术——塑造(shaping,也称为“连续逼近”)——通过分解造成稀疏奖励难以学习的行为上的不连续性(discontinuities),从本质上创建了一个密集的学习信号。例如,斯金纳(Skinner)就曾利用基于角度和位置的奖励(密集信号)来引导一只鸽子到达目标区域,最终的稀疏奖励在那里等待着它。

\subsubsection{Reward Shaping}

奖励塑造 (Reward Shaping): 可以通过势函数(potential-based)的转换,将稀疏奖励转化为密集奖励(dense reward),从而极大地促进学习过程。这种变换必须保证最优策略保持不变。

自然奖励信号(Natural Reward Signals)

1. 进化奖励信号:生存与适应度

Evolutionary Reward Signals: Survival and Fitness
	
\section{训练动力学}
	

\end{document}
